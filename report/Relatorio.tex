\documentclass[12pt,a4paper]{memoir}
% \documentclass[titlepage,12pt,a4paper]{book}

\usepackage[portuguese]{babel}
\usepackage[utf8]{inputenc}
\usepackage[T1]{fontenc}
\usepackage{makeidx}
\usepackage{xspace}
\usepackage{graphicx,color,times}
\usepackage{fancyhdr}
\usepackage{pxfonts}
\usepackage{times}
\usepackage{amssymb}
\usepackage{amsfonts}
\usepackage{amsmath}
\usepackage{latexsym}
\usepackage[printonlyused]{acronym}
\usepackage{float}
\usepackage{listings}
\usepackage{tocbibind}
\usepackage{natbib}
\usepackage[hidelinks]{hyperref}
\usepackage[T1]{fontenc}
\usepackage{titlesec, blindtext, color}
\usepackage{url}

% \renewcommand{\ttdefault}{phv}

\pagestyle{fancy}
\renewcommand{\chaptermark}[1]{\markboth{#1}{}}
\renewcommand{\sectionmark}[1]{\markright{\thesection\ #1}}
\fancyhf{} \fancyhead[LE,RO]{\bfseries\thepage}
\fancyhead[LO]{\bfseries\rightmark}
\fancyhead[RE]{\bfseries\leftmark}
\renewcommand{\headrulewidth}{0.5pt}
\renewcommand{\footrulewidth}{0pt}
\setlength{\headheight}{10pt}
\setlength{\marginparsep}{0cm}
\setlength{\marginparwidth}{0cm}
\setlength{\marginparpush}{0cm}
\addtolength{\hoffset}{-1cm}
\addtolength{\oddsidemargin}{\evensidemargin}
\addtolength{\oddsidemargin}{0cm}
\addtolength{\evensidemargin}{0.5cm}

\chapterstyle{box}

\usepackage{fix-cm}
\usepackage{fourier}
\usepackage[scaled=.92]{helvet}
\definecolor{ChapGrey}{rgb}{0.6,0.6,0.6}
\newcommand{\LargeFont}{
  \usefont{\encodingdefault}{\rmdefault}{b}{n}
  \fontsize{60}{80}\selectfont\color{ChapGrey}
  }
\makeatletter
\newcommand{\hsp}{\hspace{20pt}}

\makeatother
\chapterstyle{GreyNum}

\setcounter{tocdepth}{3}
\setsecnumdepth{subsubsection}

\renewcommand{\ttdefault}{lmtt}

% cores
\definecolor{gray75}{gray}{0.75}
\definecolor{dark}{gray}{0.25} 
\definecolor{lgray}{gray}{0.9}
\definecolor{dkblue}{rgb}{0,0.13,0.4}
\definecolor{dkgreen}{rgb}{0,0.6,0}
\definecolor{gray}{rgb}{0.5,0.5,0.5}
\definecolor{mauve}{rgb}{0.58,0,0.82}

\lstset{ %
  language=C,                    basicstyle=\footnotesize,
  numbers=none,                  numberstyle=\tiny\color{gray}, 
  stepnumber=1,                  numbersep=5pt,
  backgroundcolor=\color{white}, showspaces=false,
  showstringspaces=false,        showtabs=false,
  frame=single,                  rulecolor=\color{black},
  tabsize=2,                     captionpos=b,
  breaklines=true,               breakatwhitespace=false,
  title=\lstname,                keywordstyle=\color{blue},
  commentstyle=\color{dkgreen},  stringstyle=\color{mauve},
  escapeinside={\%*}{*)},        morekeywords={*},
  belowskip=0cm
}

\renewcommand{\lstlistingname}{Excerto de Código}
\renewcommand{\lstlistlistingname}{Lista de Excertos de Código}

\renewcommand{\today}{\day \ifcase \month \or Janeiro\or Fevereiro\or Março\or %
Abril\or Maio\or Junho\or Julho\or Agosto\or Setembro\or Outubro\or Novembro\or %
Dezembro\fi de \number \year} 

\titleformat{\chapter}[hang]{\Huge\bfseries}{\thechapter\hsp\textcolor{gray75}{/}\hsp}{0pt}{\Huge\bfseries\slshape}
\titleformat{\section}[hang]{\Large\bfseries}{\thesection\hsp\textcolor{gray75}{|}\hsp}{0pt}{\Large\bfseries}

\begin{document}

\include{capa}

\thispagestyle{empty}
\setcounter{page}{-1}

\begin{center}
\begin{Huge}
\textbf{Universidade da Beira Interior}
\end{Huge}
\end{center}

\begin{center}
\begin{Huge}
Departamento de Informática
\end{Huge}
\end{center}

\vspace{0,07cm}
\begin{figure}[!htb]
\centering
\includegraphics[width=191pt]{ubi-fe-di.png}
\end{figure}

\vspace{0.5cm}
\begin{center}
\begin{Large}
\textbf{\emph{Nome}}
\end{Large}
\end{center}


\vspace{0.5cm}
\begin{center}
\begin{normalsize}
\begin{large}
Elaborado por:
\end{large}
\end{normalsize}
\end{center}

\begin{center}
\begin{large}
\textbf{Aluno - Nº} \\
\end{large}
\end{center}

\vspace{0,5cm}
\begin{center}
\begin{normalsize}
\begin{large}
Orientador:
\end{large}
\end{normalsize}
\end{center}

\begin{center}
\begin{large}
\textbf{Professor}
\end{large}
\end{center}



\vspace{0.5cm}
\begin{center}
\begin{normalsize}
\today
\end{normalsize}
\end{center}

\clearpage{\thispagestyle{empty}\cleardoublepage}

\frontmatter

\acresetall

\include{agradecimentos}

\chapter*{Agradecimentos}
\label{chap:agradec}

Lorem ipsum

\clearpage{\thispagestyle{empty}\cleardoublepage}

\include{resumo}

\chapter*{Resumo}
\label{chap:resumo}

Lorem Ipsum

\clearpage{\thispagestyle{empty}\cleardoublepage}
\lstlistoflistings

\clearpage{\thispagestyle{empty}\cleardoublepage}

\include{acronimos}

\chapter*{Acrónimos}
\label{chap:acro}

\begin{acronym}
    \acro{LI}{Lorem Ipsum}
\end{acronym}

\clearpage{\pagestyle{empty}\cleardoublepage}

\tableofcontents

\clearpage{\thispagestyle{empty}\cleardoublepage}

\mainmatter

\include{introducao}

\chapter{Introdução}
\label{chap:intro}

\section{Enquadramento}
\label{sec:amb}

Lorem Ipsum

\section{Motivação}
\label{sec:mot}

Lorem Ipsum

\section{Objetivos}
\label{sec:obj}

Lorem Ipsum

\newpage

\section{Organização do Documento}
\label{sec:organ}

Lorem Ipsum

\begin{enumerate}
\item O primeiro capítulo -- \textbf{Introdução} -- Lorem Ipsum
\item O segundo capítulo -- \textbf{Estado-da-arte} -- Lorem Ipsum
\item O terceiro capítulo -- \textbf{Tecnologias Utilizadas} -- Lorem Ipsum
\item O quarto capítulo -- \textbf{Implementação e Testes} -- Lorem Ipsum
\item O quinto capítulo -- \textbf{Conclusões e Trabalho Futuro} -- Lorem Ipsum
\end{enumerate}

\clearpage{\thispagestyle{empty}\cleardoublepage}

\include{capitulo-2}

\chapter{Estado-da-arte}
\label{chap:estado-da-arte}

\section{Introdução}
\label{chap2:sec:intro}


\section{Estado-da-arte}
\label{chap2:sec:estado}


\section{Conclusões}
\label{chap2:sec:concs}

\clearpage{\thispagestyle{empty}\cleardoublepage}

\include{capitulo-3}

\chapter{Tecnologias e Ferramentas}
\label{chap:tecno-ferra}

\section{Introdução}
\label{chap3:sec:intro}

\section{Tecnologias Importantes}
\label{chap3:sec:tecno-imp}

\newpage

\section{Utilização}
\label{chap3:util}


\section{Conclusões}
\label{chap3:sec:concs}

\clearpage{\thispagestyle{empty}\cleardoublepage}

\include{capitulo-4}

\chapter{Implementação e Testes}
\label{chap:imp-test}

\section{Introdução}
\label{chap4:sec:intro}

\section{Dependências}
\label{chap4:sec:depend}

\newpage

\section{Detalhes da Implementação}
\label{chap4:sec:det-imp}

\begin{lstlisting}[caption=\texttt{quadruplo()} e o seu funcionamento.]
void quadruplo(int x[][1])
{
  int res[2][N];

  for (int i = 0; i < N; i++)
  {
    res[0][i] = x[i][0];
    res[1][i] = 4*x[i][0]; 
  }
  printf("O vetor onde a segunda linha e o resultado da multiplicacao por 4 da primeira e:\n");
  escrever(2, N, res);
}
\end{lstlisting}

\vspace{\baselineskip} \noindent

$$ RES  = \begin{bmatrix}
a & b & c & ...\\
4*a & 4*b & 4*c & ...
\end{bmatrix}
$$


\newpage

\noindent

\section{Testes}
\label{sec:testes}

\newpage

\section{Conclusões}
\label{chap4:sec:concs}

\clearpage{\thispagestyle{empty}\cleardoublepage}

\include{conc-trab-futuro}

\chapter{Conclusões e Trabalho Futuro}
\label{chap:conc-trab-futuro}

\section{Conclusões Principais}
\label{sec:conc-princ}


\section{Trabalho Futuro}
\label{sec:trab-futuro}

\clearpage{\thispagestyle{empty}\cleardoublepage}

\backmatter

\bibliography{bibl}
\bibliographystyle{alphadin}
\nocite{LATEXDoc}
\nocite{TutorialsPoint}
\nocite{DoxygenDoc}

\end{document}
